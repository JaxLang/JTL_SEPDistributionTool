% book example for classicthesis.sty
\documentclass[8pt,a5paper,openany,oneside,footinclude=true,headinclude=true]{scrbook} % KOMA-Script book
\usepackage[T1]{fontenc}                
\usepackage{lipsum}
\usepackage[linedheaders,parts,pdfspacing, style=arsclassica]{classicthesis} % ,manychapters
%\usepackage[osf]{libertine}
\usepackage{amsthm}
\usepackage{amsmath}
\usepackage{amssymb}
\usepackage{hyperref}
\usepackage{graphicx}


\begin{document}
%	\pagestyle{scrheadings}
%	\manualmark
%	\markboth{\spacedlowsmallcaps{\contentsname}}{\spacedlowsmallcaps{\contentsname}}
	\tableofcontents 

%	\automark[section]{chapter}
%	\renewcommand{\chaptermark}[1]{\markboth{\spacedlowsmallcaps{#1}}{\spacedlowsmallcaps{#1}}}
%	\renewcommand{\sectionmark}[1]{\markright{\thesection\enspace\spacedlowsmallcaps{#1}}}

    % use \cleardoublepage here to avoid problems with pdfbookmark
    %\cleardoublepage\part{Test Part}
    \cleardoublepage\chapter{October 14, 2025}
		\newtheoremstyle{note}% hnamei 
		{3pt}% hSpace abovei 
		{3pt}% hSpace belowi note
		{}% hBody fonti 
		{}% hIndent amounti1
		{\itshape}% hTheorem head fonti 
%		{\spacedlowsmallcaps}%
		{:}% hPunctuation after theorem headi 
		{.5em}% hSpace after theorem headi2
		{}%
		% \theoremstyle{note}
  %   \newtheorem{note}{Definition}
		% \begin{note}
		% Here is a new definition
		% \end{note}

    
		% \begin{proof}
		% Here is my proof:
		% \[
		% a^2 + b^2 = c^2 \qedhere
		% \]
		% \end{proof}    
        \textit{Attendees:} Nina Dresing, Rami Vainio, Jan Gieseler, Annamaria Fedeli, Eeli Kinnunen, Jax T. Lang.\\

        \begin{enumerate}
            \item Tools due with December deliverable right before Christmas (Except Eeli's).
            \item Meeting every week on Tuesday at 9:30 am, alternating between Annamaria and Jax. Starting with Jax on 4 November.
        \end{enumerate}

        \section{Task 5-3} 
        \href{https://seafile.utu.fi/d/9be7b991ef8a4a3588d0/files/?p=%2F05%20-%20Dresing%20WP5_Plan.pdf}{See slide 13 of Nina's WP5 plan for details of the tasks due.}

        \begin{itemize}
            \item ND wants to include SolarMACH (plots, 3D, and coordinate table).
            \item RV wants to provide pre-selected energy-species combos: 25-40 MeV.
            \item Keep the radial scaling, intercalibration, and background subtracting, but allow the user to input / change the values.
            \item Plots required: Timeseries, Gaussian gif images.
            \item Test the tool on the radio group and Farwa.
            \item Code should be almost entirely hidden, save for the user inputs and 1 main running function.
        \end{itemize}
        \textbf{Latitudinal Assessment:} Fit the ecliptic plane spacecraft then plot solar orbiter in the gaussian plots to see the offset. \\
        \textit{Jax:} Can we get a Bepi loader in time for this? We need to have 4 or more data points to fit the gaussian.

    \vspace{0.5cm}

        \textbf{Exclude:}
        \begin{enumerate}
            \item Model inputs and models.
            \item Linear fits to gaussian results.
            \item Contamination regions.
            \item Smoothing.
            \item External data.
        \end{enumerate}
        
    
    %%%%%%%%%%%%%%%%%%%%%%%%%%%%%%%%%%%%%%%%%%%%%%%%%%%%%%%%%%
    \chapter{November 4, 2025}

    \textit{Attendees:} Nina Dresing, Jan Gieseler, Jax T. Lang.\\

        \begin{enumerate}
            \item Focus on 14~MeV proton channels because they are smoother and easier to work with. Can change this later.
            \item Make the flare input optional and only consider spacecraft footprint longitude in calculations.
        \end{enumerate}

        The notebook should in the following order:
        \begin{enumerate}
            \item User inputs event details.
            \begin{itemize}
                \item \textit{Optional: Flare location for reference in plots.}
            \end{itemize}
            \item Plots and prints SolarMACH details.
            \item Load spacecraft data:
            \begin{enumerate}
                \item User confirms which Spacecraft to include. (Done as checkboxes with all of them ticked as default. If unticked, confirm that they're still included in final plot but not the fitting process.)
                \item User confirms options for channels in a dictionary with intercalibration factors. (Red box to warn users about changing the values themselves).
                \item User chooses the resampling time, default: 15 minutes.
                \item Intercalibration. User can opt out with True/False.
                \item Radial Scaling. User can opt out with True/False.
            \end{enumerate}
            \item Plot time series and allow user to define background for background subtraction.
            \item Calculate the Gaussian Fits 
            \begin{enumerate}
                \item First with \texttt{scipy.curve\_fit}, then with \texttt{scipy.ODR}, with the uncertainties.
                \item Save figures at each timestep into a subfolder as it runs.\\
                Gaussian curve left, intensity time series to the right with vertical line to indicate time step.
            \end{enumerate}
            \item Plot final time series with 3 rows of plots:
            \begin{enumerate}
                \item Intensity time series
                \item Gaussian center
                \item Gaussian width
            \end{enumerate}
            \item \textit{(Optional)} Gif of the gaussian figs saved in the sub folder.
            \item \textit{To do later:} Latitude fits with Bepi data. Try set up using manually loaded Bepi-SIXS data for now.
        \end{enumerate}

    \textbf{Deadline}: Week before Christmas, \textbf{Jax internal deadline: 1st week of December.}\\

    Jax to create a public github repo to share with others as we go.
    
    %%%%%%%%%%%%%%%%%%%%%%%%%%%%%%%%%%%%%%%%%%%%%%%%%%%%%%%%%%
    \chapter{November 18, 2025}
    \textit{Attendees:} Nina Dresing, Jan Gieseler, Rami Vainio, Jax T Lang. \\

    \begin{enumerate}
        \item Resampling should be done using the following calculations for the intensity $I$ and uncertainty $\sigma$.
        \begin{align}
            <I> &= \tfrac{1}{n} \sum_{j=1}^w I_j \\
            <\sigma> &= \tfrac{1}{n} \sqrt{\sum_{j=1}^w \sigma_j^2}
        \end{align}
        \item The order of operations should run as:
        \begin{enumerate}
            \item Load data (merge, omni, resample, etc)
            \item Background subtract
            \item Inter calibrate
            \item Radial Scale
        \end{enumerate}
        \item Radial scaling calculation is fine (although it should be done with integrals etc, this version is more than sufficient).
        \item Background Subtracting should only use the average of the background (not the avg - std deviation like I had it). The background also needs to be calculated like the equations given in the resampling point.
        \item JTL aims to be done by Friday (21 November 2025) and thereafter just polishing and testing new events.
        \item RV not joining for next meeting on 2 December 2025.
        \item Meeting on 16 December 2025 merged with JTL's quarterly update with ND and RV.
    \end{enumerate}


    %%%%%%%%%%%%%%%%%%%%%%%%%%%%%%%%%%%%%%%%%%%%%%%%%%%%%%%%%%
    \chapter{November 20, 2025}
    \textit{Attendees:} Nina Dresing, Jax T Lang. \\

    \begin{enumerate}
        \item All changes implemented and uncertainty concern is fixed (prior to 18 Nov update meeting, the uncertainty values were very high, especially for SolO). 
        \item When calculating Gaussian curves, I ensure that any points with nan values or values less than $1\times10^{-6}$ are removed. This tool code removes points with nan values and zero values. Might adapt later if need-be.
        \item If coordinate system is set to Stonyhurst then the Gaussians should be fitted from -180 to +180; if it is set to Carrington then 0 to 360.
        \item Spacecraft need to be ordered so that the maximum intensity is centered.
\begin{table}[h!]
\centering
\begin{tabular}{lllll}
\textbf{} & \textbf{SOHO} & \textbf{PSP} & \textbf{STEREO-A} & \textbf{Wind} \\ \cline{5-5} 
\textbf{x} & -150 & -90 & \multicolumn{1}{l|}{120} & \multicolumn{1}{l|}{170} \\ \cline{5-5} 
\textbf{y} & 1000 & 250 & \multicolumn{1}{l|}{350} & \multicolumn{1}{l|}{1500} \\ \cline{5-5} 
\multicolumn{5}{l}{\textbf{}} \\\hline
 & \textbf{STEREO-A} & \textbf{Wind} & \textbf{SOHO} & \textbf{PSP} \\ \cline{3-3}
\textbf{x} & \multicolumn{1}{l|}{120} & \multicolumn{1}{l|}{170} & \textit{-150 (210)} & \textit{-90 (270)} \\ \cline{3-3}
\textbf{y} & \multicolumn{1}{l|}{350} & \multicolumn{1}{l|}{1500} & 1000 & 250 \\ \cline{3-3}
\end{tabular}
\caption{Here we see in the top part of the table how the max intensity (Wind) is on the outer edge. So we adjust this by wrapping the values (from -180:180, to 0:360) and shifting the starting point for the list so that the maximum intensities are in the center with the correct order.}
\end{table}
        \item Still to add vline in time series to indicate the timestamp.
        \item Try to define axis limits based on the data (must be consistent for every run).
    \end{enumerate}



    
    %%%%%%%%%%%%%%%%%%%%%%%%%%%%%%%%%%%%%%%%%%%%%%%%%%%%%%%%%%
    \chapter{November 27, 2025}
    \textit{Attendees:} Nina Dresing, Jax T Lang. \\

    \begin{enumerate}
        \item Converted the whole tool to a class with functions running within, seems much smoother now.
        \item IC: Need to explicitly state that STA-HET is the baseline.
        \item RS: Small need-to-know info at the section, longer breakdown in 'Appendix'.
        \item RS: What if only \texttt{a} is provided and not \texttt{a $\pm$ b}?
        \item Before Gaussian Fit section: Plot old school \textit{Peak Intensity Gaussian Fit} to allow user the option to exclude a spacecraft from fitting calculations.
        \item GF: 
        \begin{itemize}
            \item (Option A) Plot the first fig, then \texttt{clf} the rest. \\ (Option B) Don't show any plots, just show end gif.\\
        \begin{verbatim}
            from IPython.display import Image
            ...
            gif = Image(filename="Gaussfits.gif")
            display(gif)
        \end{verbatim}
        \item Save the csv after dropping all lines that are just \texttt{NaN}s.
        \item Move bulk of info to appendix.
        \end{itemize}
         \item Time series plot: 
        \begin{itemize}
            \item Add xlabel: "Time \& Date"
            \item Add energy labels to (all) legends. Make the top legend 2 columns: "PSP ISOIS HET: 0.9 - 1.4 MeV protons"
            \item Explain all elements: flare start time (vertical line), flare location (horizontal)...
        \end{itemize}
        \item Add Appendix with calculation explanations.
        \item Add glossary with optional functions for the user (download the data, plot time series, adjust figure?).
        \item Remove Latitude cell.
        \item Start notebook with an image of example result.
        \item Add table of contents.
    \end{enumerate}




    
    %%%%%%%%%%%%%%%%%%%%%%%%%%%%%%%%%%%%%%%%%%%%%%%%%%%%%%%%%%
    \chapter{December 2, 2025}
    \textit{Attendees:} Jax T Lang. \\

    \begin{enumerate}
        \item 
    \end{enumerate}


    
    %%%%%%%%%%%%%%%%%%%%%%%%%%%%%%%%%%%%%%%%%%%%%%%%%%%%%%%%%%
    \chapter{December 16, 2025}
    \textit{Attendees:} Jax T Lang. \\

    \begin{enumerate}
        \item 
    \end{enumerate}


    
    %%%%%%%%%%%%%%%%%%%%%%%%%%%%%%%%%%%%%%%%%%%%%%%%%%%%%%%%%%
    \chapter{January 13, 2026}
    \textit{Attendees:} Jax T Lang. \\

    \begin{enumerate}
        \item 
    \end{enumerate}

%	\include{multiToC}

    \appendix
    \cleardoublepage\part{Appendix}
    \chapter{Calculations for Each Step}

    \section{Load the data}
    \subsection{Parker Solar Probe}
    Load using 
    \begin{verbatim}
    pspdf, psp_meta = psp_isois_load(dataset='PSP_ISOIS-EPIHI_L2-HET-RATES60', 
                                    startdate=dates[0], enddate=dates[1],
                                    path=data_path+'psp/', resample=None)  
    \end{verbatim} and the relevant columns are defined as \texttt{A\_H\_Flux\_n} and \texttt{B\_H\_Uncertainty\_n}. \\

    The bin width is then calculated using the bin label from \texttt{psp\_meta} under the title of \texttt{H\_ENERGY\_LABL} (provided as \texttt{22.6 - 40.0 MeV} so it is split to just be the values).\\

    Then we merge the channels using the weighted bin merge:
    \begin{align}
        f_{x-y} &= \frac{1}{\text{\# of channels}}\sum_{n=x}^{y} (f_n \cdot DE_n), \tag{Weighted bin merge} \label{eq:weightedbinmerge}
    \end{align} where the channels run from $x$ to $y$ and $DE$ is the bin width found in the previous step. \\

    Then we make it omni-directional using a simple nanmean function for both flux and uncertainty.\\

    Lastly, it is resampled to 15 minutes using the pandas resample function and applying the standard \texttt{.mean()} function to every column except the uncertainty which uses a root mean square equation:
    \begin{align}
        <\sigma> &= \frac{1}{n} \sqrt{\sum_{j=1}^w \sigma_j^2}, \label{eqn:rms_mean} \tag{Root Mean Square}
    \end{align} where $\sigma$ is the uncertainty.\\

    In the end, it is concatenated with the solarmach loop dataframe.
    %%%%%%%%%%%%%%%%%%%%%%%%%%%%%%%%%%%%%%%%%%%%%%%%%%%%%%%%%%%%%%
    \subsection{SOHO}

    Load using 
    \begin{verbatim}
    sohodf, soho_meta = soho_load(dataset='SOHO_ERNE-HED_L2-1MIN',
                                  startdate=dates[0], enddate=dates[1],
                                  path=data_path+'soho/', resample=None,
                                  pos_timestamp='start')  
    \end{verbatim} and the relevant columns are defined as \texttt{PH\_n} and \texttt{PHC\_n}. \\

    The bin width is then calculated using the bin label from \texttt{soho\_meta} under the title of \texttt{[channels\_dict\_df\_p]} then \texttt{lower\_E} or \texttt{upper\_E} (provided as the upper and lower limits of the bin).\\

    From here, the uncertainty needs to be calculated using:
    \begin{align}
        \sigma_n &= \frac{f_n}{counts/s} \times 1.1, \label{eqn:sohouncertainty} \tag{SOHO Uncertainty}
    \end{align} where 10\% is added for systematic errors.\\

    Then we merge the channels using the weighted bin merge from above. \\

    Then we don't need to make it omni-directional.\\

    Lastly, it is resampled to 15 minutes using the same system as above.\\

    In the end, it is concatenated with the solarmach loop dataframe.
    %%%%%%%%%%%%%%%%%%%%%%%%%%%%%%%%%%%%%%%%%%%%%%%%%%%%%%%%%%%%%%
    \subsection{Solar Orbiter}
    Load using 
    \begin{verbatim}
    solo_dfp, solo_dfe, solo_meta = epd_load(sensor='het', level='l2',
                                      startdate=dates[0], enddate=dates[1],
                                      viewing='omni', autodownload=True,
                                      pos_timestamp='start', path=data_path+'solo/')
    \end{verbatim} and the relevant columns are defined as \texttt{[('H\_Flux','H\_Flux\_n')]} and \texttt{[('H\_Uncertainty','H\_Uncertainty\_n')]}. \\

    The bin width is then calculated using the bin label from \texttt{solo\_meta} under the title of \texttt{H\_Bins\_Width}.\\

    Then we merge the channels using the weighted bin merge from above. \\

    Then we don't need to make it omni-directional.\\

    Lastly, it is resampled to 15 minutes using the same system as above.\\

    In the end, it is concatenated with the solarmach loop dataframe.
    %%%%%%%%%%%%%%%%%%%%%%%%%%%%%%%%%%%%%%%%%%%%%%%%%%%%%%%%%%%%%%
    \subsection{STEREO-A}
    Load using 
    \begin{verbatim}
    stadf, sta_meta = stereo_load(instrument='HET', spacecraft='ahead',
                                  startdate=dates[0], enddate=dates[1],
                                  path=data_path+'stereo/', resample=None,
                                  pos_timestamp='start')
    \end{verbatim} and the relevant columns are defined as \texttt{Proton\_Flux\_n} and \texttt{Proton\_Sigma\_n}. \\

    The bin width is then calculated using the bin label from \texttt{sta\_meta} under the title of \texttt{[channels\_dict\_df\_p]} then \texttt{lower\_E} or \texttt{upper\_E} (provided as the upper and lower limits of the bin).\\

    Then we merge the channels using the weighted bin merge from above. \\

    Then we don't need to make it omni-directional.\\

    Lastly, it is resampled to 15 minutes using the same system as above.\\

    In the end, it is concatenated with the solarmach loop dataframe.
    %%%%%%%%%%%%%%%%%%%%%%%%%%%%%%%%%%%%%%%%%%%%%%%%%%%%%%%%%%%%%%

    
    \section{Background Subtraction}
    \begin{itemize}
        \item The user provides a window of two datetime values in a list.\\
        A figure is plotted to confirm the window with the user before continuing into the function.
        \item Separate lists are made of just the flux (or uncertainty) values in the given window.
        \item The background average is calculated from that (the flux is standard average, but the uncertainty uses \ref{eqn:rms_mean}).
        \item The original column is then adjusted based on this value:\\
        \begin{itemize}
            \item $f = f_0 - f_{bg}$
            \item $\sigma = \sqrt{\sigma_0^2 + \sigma_{bg}^2}$
        \end{itemize}
        \item Any negative values are replaced with NaNs.
    \end{itemize}
    
    
    

    \section{Intercalibration}
    Each spacecraft has its own IC factor and both the flux and uncertainty columns are simply multiplied by this factor.


    \section{Radial Scaling}
    Using the formula presented in FarwaEA2025 and the scaling values from LarioEA2006 for protons and from Rodriguez-GarciaEA2023 for ($\sim$100 keV) electrons. The formula we use is then:
    \begin{align}
        I_{(1~au)} &= I \cdot R^{a \pm b}, \label{eqn:radialscaling} \tag{Radial Scaling}
    \end{align} where $a\pm b$ are the scaling factors.

    For this tool which uses 11-16 MeV protons, we use the values from LarioEA2006 for 4-13 MeV proton results $2.14 \pm 0.26$ from Table 3 (the other range is 27-37 MeV protons which results in $1.97 \pm 0.27$).\\

    The uncertainty is more complicated because we already have uncertainty values:
    \begin{align}
        \text{1. }& \; I_+ = I \cdot R^{a+b}; \implies \Delta I_+ = |I_{(1~au)} - I_+| \\
        & \; I_- = I \cdot R^{a-b}; \implies \Delta I_- = |I_{(1~au)} - I_-| \\
        & \; \hookrightarrow \; \Delta I_{max limit} = max(\Delta I_+, \; \Delta I_-) \\
        \text{2. }& \; \sigma_{(1~au)} = \sigma \cdot R^a\\
        \text{3. }& \; \therefore  \sigma_{final} = \sqrt{(\Delta I_{max limit})^2 + (\sigma_{(1~au)})^2}
    \end{align}

    This has to be computed for \textit{each} timestep.


    \section{Gaussian Fitting}
    
    

\end{document}
% book example for classicthesis.sty
\documentclass[8pt,a5paper,openany,oneside,footinclude=true,headinclude=true]{scrbook} % KOMA-Script book
\usepackage[T1]{fontenc}                
\usepackage{lipsum}
\usepackage[linedheaders,parts,pdfspacing, style=arsclassica]{classicthesis} % ,manychapters
%\usepackage[osf]{libertine}
\usepackage{amsthm}
\usepackage{hyperref}


\begin{document}
%	\pagestyle{scrheadings}
%	\manualmark
%	\markboth{\spacedlowsmallcaps{\contentsname}}{\spacedlowsmallcaps{\contentsname}}
	\tableofcontents 

%	\automark[section]{chapter}
%	\renewcommand{\chaptermark}[1]{\markboth{\spacedlowsmallcaps{#1}}{\spacedlowsmallcaps{#1}}}
%	\renewcommand{\sectionmark}[1]{\markright{\thesection\enspace\spacedlowsmallcaps{#1}}}

    % use \cleardoublepage here to avoid problems with pdfbookmark
    %\cleardoublepage\part{Test Part}
    \cleardoublepage\chapter{October 14, 2025}
		\newtheoremstyle{note}% hnamei 
		{3pt}% hSpace abovei 
		{3pt}% hSpace belowi note
		{}% hBody fonti 
		{}% hIndent amounti1
		{\itshape}% hTheorem head fonti 
%		{\spacedlowsmallcaps}%
		{:}% hPunctuation after theorem headi 
		{.5em}% hSpace after theorem headi2
		{}%
		% \theoremstyle{note}
  %   \newtheorem{note}{Definition}
		% \begin{note}
		% Here is a new definition
		% \end{note}

    
		% \begin{proof}
		% Here is my proof:
		% \[
		% a^2 + b^2 = c^2 \qedhere
		% \]
		% \end{proof}    
        \textit{Attendees:} Nina Dresing, Rami Vainio, Jan Gieseler, Annamaria Fedeli, Eeli Kinnunen, Jax T. Lang.\\

        \begin{enumerate}
            \item Tools due with December deliverable right before Christmas (Except Eeli's).
            \item Meeting every week on Tuesday at 9:30 am, alternating between Annamaria and Jax. Starting with Jax on 4 November.
        \end{enumerate}

        \section{Task 5-3} 
        \href{https://seafile.utu.fi/d/9be7b991ef8a4a3588d0/files/?p=%2F05%20-%20Dresing%20WP5_Plan.pdf}{See slide 13 of Nina's WP5 plan for details of the tasks due.}

        \begin{itemize}
            \item ND wants to include SolarMACH (plots, 3D, and coordinate table).
            \item RV wants to provide pre-selected energy-species combos: 25-40 MeV.
            \item Keep the radial scaling, intercalibration, and background subtracting, but allow the user to input / change the values.
            \item Plots required: Timeseries, Gaussian gif images.
            \item Test the tool on the radio group and Farwa.
            \item Code should be almost entirely hidden, save for the user inputs and 1 main running function.
        \end{itemize}
        \textbf{Latitudinal Assessment:} Fit the ecliptic plane spacecraft then plot solar orbiter in the gaussian plots to see the offset. \\
        \textit{Jax:} Can we get a Bepi loader in time for this? We need to have 4 or more data points to fit the gaussian.

    \vspace{0.5cm}

        \textbf{Exclude:}
        \begin{enumerate}
            \item Model inputs and models.
            \item Linear fits to gaussian results.
            \item Contamination regions.
            \item Smoothing.
            \item External data.
        \end{enumerate}
        
    \cleardoublepage
    \chapter{November 4, 2025}

    \textit{Attendees:} Nina Dresing, Jan Gieseler, Jax T. Lang.\\

        \begin{enumerate}
            \item Focus on 14~MeV proton channels because they are smoother and easier to work with. Can change this later.
            \item Make the flare input optional and only consider spacecraft footprint longitude in calculations.
        \end{enumerate}

        The notebook should in the following order:
        \begin{enumerate}
            \item User inputs event details.
            \begin{itemize}
                \item \textit{Optional: Flare location for reference in plots.}
            \end{itemize}
            \item Plots and prints SolarMACH details.
            \item Load spacecraft data:
            \begin{enumerate}
                \item User confirms which Spacecraft to include. (Done as checkboxes with all of them ticked as default. If unticked, confirm that they're still included in final plot but not the fitting process.)
                \item User confirms options for channels in a dictionary with intercalibration factors. (Red box to warn users about changing the values themselves).
                \item User chooses the resampling time, default: 15 minutes.
                \item Intercalibration. User can opt out with True/False.
                \item Radial Scaling. User can opt out with True/False.
            \end{enumerate}
            \item Plot time series and allow user to define background for background subtraction.
            \item Calculate the Gaussian Fits 
            \begin{enumerate}
                \item First with \texttt{scipy.curve\_fit}, then with \texttt{scipy.ODR}, with the uncertainties.
                \item Save figures at each timestep into a subfolder as it runs.\\
                Gaussian curve left, intensity time series to the right with vertical line to indicate time step.
            \end{enumerate}
            \item Plot final time series with 3 rows of plots:
            \begin{enumerate}
                \item Intensity time series
                \item Gaussian center
                \item Gaussian width
            \end{enumerate}
            \item \textit{(Optional)} Gif of the gaussian figs saved in the sub folder.
            \item \textit{To do later:} Latitude fits with Bepi data. Try set up using manually loaded Bepi-SIXS data for now.
        \end{enumerate}

    \textbf{Deadline}: Week before Christmas, \textbf{Jax internal deadline: 1st week of December.}\\

    Jax to create a public github repo to share with others as we go.
    
    
    \chapter{Next Chapter}
    \textit{Attendees:} Jax T Lang. \\

    \begin{enumerate}
        \item 
    \end{enumerate}

%	\include{multiToC}

    % \appendix
    % \cleardoublepage\part{Appendix}
    % \chapter{Appendix Chapter}
    % \lipsum[1]
    
    % \section{A Section}
    % \lipsum[1]

\end{document}